\section{Docker installation config files} \label{Dicf}
\subsection{.env-File:}
\begin{lstlisting}[language=bash,breaklines=true]
	COMPOSE_PROJECT_NAME=pelias 
	DATA_DIR=/data/pelias-docker-compose/ 
	DOCKER_USER=1003 
	ENABLE_GEONAMES=true
	DATA_DIR: directory where Pelias will store downloaded data. Also used to build its other services. 
	DOCKER_USER = All Pelias processes in containers run as non-root users. This user ID will be used for accessing files on the host filesystem in DATA_DIR.
\end{lstlisting}

\subsection{Elasticsearch.yml-File:}
\begin{lstlisting}[language=bash,breaklines=true]
	network.host: 0.0.0.0 
	bootstrap.memory_lock: true 
	indices.breaker.fielddata.limit: 85% 
	indices.fielddata.cache.size: 75% 
	thread_pool.bulk.queue_size: 500 
	thread_pool.index.queue_size: 1000
\end{lstlisting}

\subsection{pelias.json-File:}
\begin{lstlisting}[language=json,breaklines=true]
{
  "logger": {
    "level": "info",
    "timestamp": false
  },
  "esclient": {
    "apiVersion": "2.4",
    "hosts": [
      {
        "host": "elasticsearch"
      }
    ]
  },
  "elasticsearch": {
    "settings": {
      "index": {
        "refresh_interval": "10s",
        "number_of_replicas": "0",
        "number_of_shards": "5"
      }
    }
  },
  "acceptance-tests": {
    "endpoints": {
      "docker": "http://api:4000/v1/"
    }
  },
  "api": {
    "targets": {
      "canonical_sources": [
        "whosonfirst",
        "openstreetmap",
        "openaddresses",
        "geonames",
        "geonamesandpostcodeinfo"
      ],
      "layers_by_source": {
        "geonames2d": [
          "country",
          "postalcode",
          "source",
          "layer"
        ],
        "geonamesandpostcodeinfo": [
          "name",
          "source",
          "layer",
          "lat",
          "lon",
          "country",
          "postalcode"
        ],
        "openstreetmap": [
          "address",
          "venue",
          "street"
        ],
        "openaddresses": [
          "address"
        ],
        "geonames": [
          "country",
          "macroregion",
          "region",
          "county",
          "localadmin",
          "locality",
          "borough",
          "neighbourhood",
          "venue",
          "postalcode"
        ],
        "whosonfirst": [
          "continent",
          "empire",
          "country",
          "dependency",
          "macroregion",
          "region",
          "locality",
          "localadmin",
          "macrocounty",
          "county",
          "macrohood",
          "borough",
          "neighbourhood",
          "microhood",
          "disputed",
          "venue",
          "postalcode",
          "continent",
          "ocean",
          "marinearea"
        ]
      },
      "source_aliases": {
        "osm": [
          "openstreetmap"
        ],
        "oa": [
          "openaddresses"
        ],
        "gn": [
          "geonames"
        ],
        "wof": [
          "whosonfirst"
        ],
        "2dpg": [
          "geonames2d"
        ]
      }
    },
    "textAnalyzer": "libpostal",
    "services": {
      "pip": {
        "url": "http://pip:4200"
      },
      "libpostal": {
        "url": "http://libpostal:4400"
      },
      "placeholder": {
        "url": "http://placeholder:4100"
      },
      "interpolation": {
        "url": "http://interpolation:4300"
      }
    },
    "defaultParameters": {
      "focus.point.lat": 48.88,
      "focus.point.lon": 2.32
    }
  },
  "imports": {
    "adminLookup": {
      "enabled": true
    },
    "geonames": {
      "datapath": "/data/geonames",
      "countryCode": "ALL",
      "sourceURL": "http://download.geonames.org/export/dump/"
    },
    "openstreetmap": {
      "download": [
        {
          "sourceURL": "https://download.geofabrik.de/europe-latest.osm.pbf"
        }
      ],
      "leveldbpath": "/tmp",
      "datapath": "/data/openstreetmap",
      "import": [
        {
          "filename": "europe-latest.osm.pbf"
        }
      ]
    },
    "openaddresses": {
      "datapath": "/data/openaddresses",
      "files": [
      ]
    },
    "polyline": {
      "datapath": "/data/polylines",
      "files": [
        "europe.polyline"
      ]
    },
    "whosonfirst": {
      "datapath": "/data/whosonfirst",
      "sqlite": true,
      "importVenues": false,
      "importPostalcodes": true
    },
    "csv": {
      "datapath": "/data/geonamesandpostcodeinfo/",
      "files": [],
      "download": []
    }
  }
}
\end{lstlisting}

\subsection{docker-compose.yml-File:}
\begin{lstlisting}[language=json,breaklines=true]
version: '3'
networks:
  default:
    driver: bridge
services:
  libpostal:
    image: pelias/libpostal-service
    container_name: pelias_libpostal
    user: "${DOCKER_USER}"
    restart: always
    ports: [ "4400:4400" ]
  schema:
    image: pelias/schema:master
    container_name: pelias_schema
    user: "${DOCKER_USER}"
    volumes:
      - "./pelias.json:/code/pelias.json"
  api:
    image: pelias/api:master
    container_name: pelias_api
    user: "${DOCKER_USER}"
    restart: always
    environment: [ "PORT=4000" ]
    ports: [ "4000:4000" ]
    volumes:
      - "./pelias.json:/code/pelias.json"
  placeholder:
    image: pelias/placeholder:master
    container_name: pelias_placeholder
    user: "${DOCKER_USER}"
    restart: always
    environment: [ "PORT=4100" ]
    ports: [ "4100:4100" ]
    volumes:
      - "./pelias.json:/code/pelias.json"
      - "${DATA_DIR}:/data"
      - "./blacklist/:/data/blacklist"
  whosonfirst:
    image: pelias/whosonfirst:master
    container_name: pelias_whosonfirst
    user: "${DOCKER_USER}"
    volumes:
      - "./pelias.json:/code/pelias.json"
      - "${DATA_DIR}:/data"
      - "./blacklist/:/data/blacklist"
  openstreetmap:
    image: pelias/openstreetmap:relations_bugfix-2019-04-25-cc778095371c142147e31249947a3b43fb57d46d
    container_name: pelias_openstreetmap
    user: "${DOCKER_USER}"
    volumes:
      - "./pelias.json:/code/pelias.json"
      - "${DATA_DIR}:/data"
      - "./blacklist/:/data/blacklist"
  openaddresses:
    image: pelias/openaddresses:master
    container_name: pelias_openaddresses
    user: "${DOCKER_USER}"
    volumes:
      - "./pelias.json:/code/pelias.json"
      - "${DATA_DIR}:/data"
      - "./blacklist/:/data/blacklist"
  geonames:
    image: pelias/geonames:master
    container_name: pelias_geonames
    user: "${DOCKER_USER}"
    volumes:
      - "./pelias.json:/code/pelias.json"
      - "${DATA_DIR}:/data"
      - "./blacklist/:/data/blacklist"
  csv-importer:
    image: pelias/csv-importer:master
    container_name: pelias_csv_importer
    user: "${DOCKER_USER}"
    volumes:
      - "./pelias.json:/code/pelias.json"
      - "${DATA_DIR}:/data"
      - "./blacklist/:/data/blacklist"
  transit:
    image: pelias/transit:master
    container_name: pelias_transit
    user: "${DOCKER_USER}"
    volumes:
      - "./pelias.json:/code/pelias.json"
      - "${DATA_DIR}:/data"
  polylines:
    image: pelias/polylines:master
    container_name: pelias_polylines
    user: "${DOCKER_USER}"
    volumes:
      - "./pelias.json:/code/pelias.json"
      - "${DATA_DIR}:/data"
  interpolation:
    image: pelias/interpolation:master
    container_name: pelias_interpolation
    user: "${DOCKER_USER}"
    restart: always
    environment: [ "PORT=4300" ]
    ports: [ "4300:4300" ]
    volumes:
      - "./pelias.json:/code/pelias.json"
      - "${DATA_DIR}:/data"
  pip:
    image: pelias/pip-service:master
    container_name: pelias_pip-service
    user: "${DOCKER_USER}"
    restart: always
    environment: [ "PORT=4200" ]
    ports: [ "4200:4200" ]
    volumes:
      - "./pelias.json:/code/pelias.json"
      - "${DATA_DIR}:/data"
  elasticsearch:
    image: pelias/elasticsearch
    container_name: pelias_elasticsearch
    restart: always
    environment: [ "ES_JAVA_OPTS=-Xmx12g" ]
    ports: [ "9200:9200", "9300:9300" ]
    volumes:
      - "./elasticsearch.yml:/usr/share/elasticsearch/config/elasticsearch.yml:ro"
      - "${DATA_DIR}/elasticsearch:/usr/share/elasticsearch/data"
    ulimits:
      memlock:
        soft: -1
        hard: -1
      nofile:
        soft: 65536
        hard: 65536
    cap_add: [ "IPC_LOCK" ]
  fuzzy-tester:
    image: pelias/fuzzy-tester:master
    container_name: pelias_fuzzy_tester
    user: "${DOCKER_USER}"
    restart: "no"
    command: "--help"
    volumes:
      - "./pelias.json:/code/pelias.json"
      - "./test_cases:/code/pelias/fuzzy-tester/test_cases"
\end{lstlisting}

\section{Pelias from scratch instllation guide} \label{Pfsig}
\begin{lstlisting}[language=bash,breaklines=true]
Dependecies:
1. Node.js:
Version 8 or newer required, version 10 recommended 
Ubuntu or Debian:
Node.js v11.x:
# Using Ubuntu
curl -sL https://deb.nodesource.com/setup_11.x | sudo -E bash -
sudo apt-get install -y nodejs
# Using Debian, as root
curl -sL https://deb.nodesource.com/setup_11.x | bash -
apt-get install -y nodejs
Node.js v10.x:
# Using Ubuntu
curl -sL https://deb.nodesource.com/setup_10.x | sudo -E bash -
sudo apt-get install -y nodejs
# Using Debian, as root
curl -sL https://deb.nodesource.com/setup_10.x | bash -
apt-get install -y nodejs
CentOS:
NodeJS 11.x
curl -sL https://rpm.nodesource.com/setup_11.x | bash -
NodeJS 10.x
curl -sL https://rpm.nodesource.com/setup_10.x | bash -
	Optional: install build tools
	To compile and install native addons from npm you may also need 
	to install build tools:
yum install gcc-c++ make
# or: yum groupinstall 'Development Tools'
2. Elasticsearch:
Version 2.4 or 5.6
wget -qO - https://artifacts.elastic.co/GPG-KEY-elasticsearch | sudo apt-key add -
echo "deb https://artifacts.elastic.co/packages/5.x/apt stable main" | sudo tee -a /etc/apt/sources.list.d/elastic-5.x.list
sudo apt update && sudo apt upgrade
sudo apt install apt-transport-https uuid-runtime pwgen openjdk-8-jre-headless
sudo apt-get update
sudo apt update
sudo apt install elasticsearch
mkdir /elasticsearch
mkdir /elasticsearch/data
mkdir /elasticsearch/logs
make sure elasticsearch has rights to access the folders:
sudo chown -R elasticsearch:elasticsearch /elasticsearch
if you get Errors like:
	ERROR Null object returned for RollingFile in Appenders
that most likely means that elasticsearch doesn't have permissions to access the logs and data folders.
After the installation, a default configuration file will be populated to 
/etc/elasticsearch/elasticsearch.yml
Most lines are commented out, edit the file to tweak and tune the configuration.
E.g, you can set correct cluster name for your applications:
cluster.name: my-application
Recommended Settings:
cluster.name: pelias-el-search
path.data: /elasticsearch/data
path.logs: /elasticsearch/logs
network.host: 127.0.0.1
http.port: 9200
Note that the default minimum memory set for JVM is 2gb, if your server has small memory size, change this value:
$ sudo vim /etc/elasticsearch/jvm.options
Change:
-Xms2g
-Xmx2g
And set your values for minimum and maximum memory allocation. E.g to set values to 512mb of ram, use:
-Xms512m
-Xmx512m
After you have modified the configuration, you can start Elasticsearch:
$ sudo systemctl daemon-reload
$ sudo systemctl enable elasticsearch.service
$ sudo systemctl restart elasticsearch.service
Check status:
$ sudo systemctl status elasticsearch.service 
 elasticsearch.service - Elasticsearch
Loaded: loaded (/usr/lib/systemd/system/elasticsearch.service; disabled; vendor preset: enabled)
Active: active (running) since Sun 2019-05-01 10:39:54 UTC; 18s ago
Docs: http://www.elastic.co
Process: 14314 ExecStartPre=/usr/share/elasticsearch/bin/elasticsearch-systemd-pre-exec (code=exited, status=0/SUCCESS)
Main PID: 14325 (java)
Tasks: 38 (limit: 2362)
CGroup: /system.slice/elasticsearch.service
14325 /usr/bin/java -Xms512m -Xmx512m -XX:+UseConcMarkSweepGC -XX:CMSInitiatingOccupancyFraction=75 -XX:+UseCMSInitiatingOccupancyOnly -X
That's all for the installation of Elasticsearch 5.x on Ubuntu 18.04 LTS (Bionic Beaver) Linux.
3. SQLite:
Version 3.11 or newer
sudo apt-get update
sudo apt-get install sqlite3
sqlite3 --version
sudo apt-get install sqlitebrowser
4.Libpostal: 
sudo apt-get install curl autoconf automake libtool pkg-config
cd /
git clone https://github.com/openvenues/libpostal
cd libpostal
./bootstrap.sh
./configure --datadir=[...some dir with a few GB of space...]
make -j4
sudo make install

# On Linux it's probably a good idea to run
sudo ldconfig
Pelias: 
Data:
Download all the rawdata you need and copy it into respective folders such as
"/data/rawdata/whosonfirst", 
"/data/rawdata/geonames", 
"/data/rawdata/openadresses", 
"/data/rawdata/openstreetmap", 
"/data/rawdata/polylines"
Pelias:
mkdir /pelias
cd /pelias
for repository in schema whosonfirst geonames openaddresses openstreetmap polylines api placeholder interpolation pip-service; do
	git clone https://github.com/pelias/${repository}.git # clone from Github
	pushd $repository > /dev/null                         # switch into importer directory
	npm install                                           # install npm dependencies
	popd > /dev/null                                      # return to code directory
done
Create a file "config.json" in the home directory (~/) of the pelias user. Paste following into pelias.json:
{
  "esclient": {
    "apiVersion": "5.6",
    "keepAlive": true,
    "requestTimeout": "120000",
    "hosts": [{
      "env": "development",
      "protocol": "http",
      "host": "localhost",
      "port": 9200
    }],
    "log": [{
      "type": "stdio",
      "json": false,
      "level": [ "error", "warning" ]
    }]
  },
  "elasticsearch": {
    "settings": {
      "index": {
        "number_of_replicas": "0",
        "number_of_shards": "5",
        "refresh_interval": "1m"
      }
    }
  },
  "interpolation": {
    "client": {
      "adapter": "null"
    }
  },
  "dbclient": {
    "statFrequency": 10000
  },
  "api": {
    "accessLog": "common",
    "textAnalyzer": "libpostal",
    "host": "http://pelias.mapzen.com/",
    "indexName": "pelias",
    "version": "1.0",
    "targets": {
      "auto_discover": false,
      "canonical_sources": ["whosonfirst", "openstreetmap", "openaddresses", "geonames"],
      "layers_by_source": {
        "openstreetmap": [ "address", "venue", "street" ],
        "openaddresses": [ "address" ],
        "geonames": [
          "country", "macroregion", "region", "county", "localadmin", "locality", "borough",
          "neighbourhood", "venue"
        ],
        "whosonfirst": [
          "continent", "empire", "country", "dependency", "macroregion", "region", "locality",
          "localadmin", "macrocounty", "county", "macrohood", "borough", "neighbourhood",
          "microhood", "disputed", "venue", "postalcode", "continent", "ocean", "marinearea"
        ]
      },
      "source_aliases": {
        "osm": [ "openstreetmap" ],
        "oa":  [ "openaddresses" ],
        "gn":  [ "geonames" ],
        "wof": [ "whosonfirst" ]
      },
      "layer_aliases": {
        "coarse": [
          "continent", "empire", "country", "dependency", "macroregion", "region", "locality",
          "localadmin", "macrocounty", "county", "macrohood", "borough", "neighbourhood",
          "microhood", "disputed", "postalcode", "continent", "ocean", "marinearea"
        ]
      }
    }
  },
  "schema": {
    "indexName": "pelias"
  },
  "logger": {
    "level": "debug",
    "timestamp": true,
    "colorize": true
  },
  "acceptance-tests": {
    "endpoints": {
      "local": "http://localhost:3100/v1/",
      "dev-cached": "http://pelias.dev.mapzen.com.global.prod.fastly.net/v1/",
      "dev": "http://pelias.dev.mapzen.com/v1/",
      "prod": "http://search.mapzen.com/v1/",
      "prod-uncached": "http://pelias.mapzen.com/v1/",
      "prodbuild": "http://pelias.prodbuild.mapzen.com/v1/"
    }
  },
  "imports": {
    "adminLookup": {
      "enabled": true,
      "maxConcurrentRequests": 100,
      "usePostalCities": false
    },
    "blacklist": {
      "files": []
    },
    "csv": {
    },
    "geonames": {
      "datapath": "/data/rawdata/geonames",
      "countryCode": "ALL"
    },
    "openstreetmap": {
      "datapath": "/data/rawdata/openstreetmaps",
      "leveldbpath": "/tmp",
      "import": [{
        "filename": "europe-latest.osm.pbf"
      }]
    },
    "openaddresses": {
      "datapath": "/data/rawdata/openaddresses",
      "files": []
    },
    "polyline": {
      "datapath": "/data/rawdata/polylines",
      "files": []
    },
    "whosonfirst": {
      "datapath": "/data/rawdata/whosonfirst",
      "importVenues": false
    }
  }
}
Add the pelias.json to the PATH variable by adding the followinbg line at the bottom of the .bashrc-file:
export PATH=$PATH:PELIAS_CONFIG=/home/<username>/pelias.config
Set up Elasticsearch Schema:
cd /pelias/schema  # assuming you have just run the bash snippet to 			  download the repos from earlier
./bin/create_index
Run the Importers:
For each importer (openadresses, openstreetmaps, whosonfirst, geonames and polylines) navigate into their folders under 
cd /pelias/<importer_folder>
and execute the importer via 
npm start
In case you want to delete the imported data and restart from import phase, run the following:
# !! WARNING: this will remove all your data from pelias!!
node scripts/drop_index.js    #it will ask for confirmation first
./bin/create_index
Install and start the pelias services:
Add the following to the bottom of the pelias.json file in your home directory. This will tell the pelias API to use all the services running locally and on their default ports.
{
  "api": {
    "services": {
      "placeholder": {
        "url": "http://localhost:3000"
      },
      "libpostal": {
        "url": "http://localhost:8080"
      },
      "pip": {
        "url": "http://localhost:3102"
      },
      "interpolation": {
        "url": "http://localhost:3000"
      }
    }
  }
}
Start the pelias API:
To start the pelias API navigate into folder
cd /pelias/api
and execute
npm start
Geocoding with pelias:
Pelias should now be up and running and will respond to your queries.
For a quick check, a request to http://localhost:3100 should display a link to the documentation for handy reference.
Here are some queries to try:
http://localhost:3100/v1/search?text=london: a search for the city of London.
http://localhost:3100/v1/autocomplete?text=londo: another query for London, but using the autocomplete endpoint which supports partial matches and is intended to be sent queries as a user types (note the query is for londo but London is returned)
http://localhost:3100/v1/reverse?point.lon=-73.986027&point.lat=40.748517: a reverse geocode for results near the Empire State Building in New York City.
\end{lstlisting}